% Vimtex is great, but damn, running LaTeX locally is such a pain

%%%%%%%%%%%%%%%%%%%%%%%%%%%%%%%%%%%%%%%%%%%%%%%%%%%%%%%%%%%%%%%%%%%%%%%%%%%%%%%%
%                                  Header {{{                                  %
%%%%%%%%%%%%%%%%%%%%%%%%%%%%%%%%%%%%%%%%%%%%%%%%%%%%%%%%%%%%%%%%%%%%%%%%%%%%%%%%
\documentclass[a4paper, 10.5pt]{article}
% \IEEEoverridecommandlockouts
% The preceding line is only needed to identify funding in the first footnote. If that is unneeded, please comment it out.
% \usepackage{cite}
\usepackage[portuguese]{babel}
\usepackage{amsmath,amssymb,amsfonts}
\usepackage{algorithmic}
\usepackage{graphicx}
\usepackage{textcomp}
\usepackage{xcolor}
\usepackage[breaklinks]{hyperref}
\usepackage{microtype}
\usepackage{natbib}
\usepackage{listings}
\usepackage{todonotes}
\usepackage{indentfirst}

\def\BibTeX{{\rm B\kern-.05em{\sc i\kern-.025em b}\kern-.08em
    T\kern-.1667em\lower.7ex\hbox{E}\kern-.125emX}}
\begin{document}

\providecommand{\keywords}[1]
{
  \small	
  \textbf{\textit{Palavras chave---}} #1
}

\title{Implementação de Assinatura Digital em Python}

\author{Leonardo Alves Riether \\
    \textit{Dep. Ciência da Computação} \\
    \textit{Universidade de Brasilia}\\
    Brasília, Brasil \\
    190032413@aluno.unb.br
}

\maketitle

% }}}

%%%%%%%%%%%%%%%%%%%%%%%%%%%%%%%%%%%%%%%%%%%%%%%%%%%%%%%%%%%%%%%%%%%%%%%%%%%%%%%%
%                                 Abstract {{{                                 %
%%%%%%%%%%%%%%%%%%%%%%%%%%%%%%%%%%%%%%%%%%%%%%%%%%%%%%%%%%%%%%%%%%%%%%%%%%%%%%%%
\begin{abstract}
    São apresentados os algoritmos e técnicas utilizadas na implementação de um
    gerador e verificador de assinaturas digitais. Foi escrito um programa em
    Python que realiza essas tarefas e, mais especificamente, implementa
    criptografia RSA, AES modo CTR, OAEP e utiliza uma biblioteca para geração
    de hashes SHA3. Decisões de projeto e detalhes específicos desta
    implementação também são explicados.

\end{abstract}

\keywords{Criptografia, AES, RSA, OAEP, SHA3, Assinatura, Digital}
% }}}

%%%%%%%%%%%%%%%%%%%%%%%%%%%%%%%%%%%%%%%%%%%%%%%%%%%%%%%%%%%%%%%%%%%%%%%%%%%%%%%%
%                                Relatório {{{                                 %
%%%%%%%%%%%%%%%%%%%%%%%%%%%%%%%%%%%%%%%%%%%%%%%%%%%%%%%%%%%%%%%%%%%%%%%%%%%%%%%%

\section{Introdução} % {{{
\label{sec:visao-geral}
    Neste trabalho foi implementado um gerador e verificador de assinaturas
    digitais em Python 3, que possibilita o envio seguro\footnote{Até certo
    ponto. Em um ambiente de produção, é recomendado o uso de bibliotecas de
criptografia mais bem testadas} de mensagens, assinadas com uma chave privada
RSA e cifradas com AES.

    O gerador de assinaturas \verb|sign.py| executa os seguintes passos:

    \begin{enumerate}
        \item Dada uma mensagem a ser assinada
        \item Gera as chaves do AES e do RSA
        \item Cifra a mensagem com AES
        \item Gera o hash SHA3 de 512 bits da mensagem original
        \item Processa o bloco $\texttt{hash} | \texttt{chave do aes} | \texttt{nonce} $
            com OAEP, onde $|$ é a operação de concatenação.
        \item Cifra o resultado com a chave privada do RSA
        \item Escreve a mensagem cifrada com AES na saída padrão e a chave
            pública em um arquivo de chave, ambos codificados em base64
    \end{enumerate}

    O procedimento de verificação em \verb|verify.py| faz o processo contrário e
    verifica a integridade da mensagem, seguindo os passos

    \begin{enumerate}
        \item Dada uma mensagem criptografada e a chave pública do RSA
        \item Decodifica o bloco $\texttt{hash} | \texttt{chave do aes} | \texttt{nonce} $
            com a chave pública, passando ele pelo OAEP.
        \item Decodifica a mensagem cifrada com AES
        \item Gera o hash SHA3 de 512 bits da mensagem decifrada
        \item Compara o hash da mensagem decifrada com o hash esperado para
            certificação de integridade
    \end{enumerate}
% }}}

\section{Implementação} % {{{
\label{sec:implementacao}
    A implementação do algoritmo descrito em \ref{sec:visao-geral} foi realizada
    em Python e testada em CPython 3.10.6. Supomos que versões anteriores da linguagem
    também podem ser utilizadas, mas não foram realizados testes em relação a
    isso.

    Para execução do programa, são recomendados os seguintes passos:
    \begin{enumerate}
        \item Criação de um ambiente virtual, com o comando \verb|python3 -m venv venv|
        \item Ativação do ambiente, com \verb|source venv/bin/activate|
        \item Instalação de dependências do projeto, por meio de \\ \verb|pip install -r requirements.txt|
        \item Execução do algoritmo de assinatura de um arquivo
            \verb|input.txt|: \\ \verb|python3 -m sign.sign -i input.txt -k key.pub > output.aes|
        \item Execução do algoritmo de verificação de assinatura: \\
            \verb|python3 -m sign.verify -i output.aes -k key.pub|
    \end{enumerate}

    As opções disponíveis podem ser vistas por meio do comando \verb|python3 -m sign.sign --help| ou \verb|python3 -m sign.verify --help|.
    Por padrão, o programa imprime os resultados na tela com cores. Para
    desabilitar essa função, basta definir a variável de ambiente
    \verb|NO_COLOR|, por exemplo: \verb|NO_COLOR=1 python3 -m sign.sign|.

\subsection{Arquivos Importantes} % {{{
\label{sec:arquivos}
    Todos os scripts em Python se encontram no diretório \verb|sign/|. Alguns
    arquivos importantes nele são:
    \begin{itemize}
        \item \verb|sign.py|: módulo principal do gerador de assinaturas
        \item \verb|verify.py|: módulo prinpical do verificador de assinaturas
        \item \verb|aes.py|, \verb|rsa.py|, \verb|oaep.py|,
            \verb|miller_rabin.py|: implementação dos respectivos algoritmos
        \item \verb|arith.py|: funções auxiliares de aritmética modular, como
            Euclides estendido e exponenciação rápida
        \item \verb|galois_field|: implementação de operações envolvendo
            $GF(2^8)$
    \end{itemize}
% Arquivos Importantes }}}

% }}}

\section{SHA3} % {{{
\label{sec:sha3}
O hash SHA3 não foi implementado do zero. Em vez disso, foi empregada a biblioteca
\textit{pysha3} \citep{pysha3}, que produz o hash de acordo com especificações.
% SHA3 }}}

\section{Rivest–Shamir–Adleman (RSA)} % {{{
O pacote $\texttt{hash} | \texttt{chave do aes} | \texttt{nonce} $ é cifrado com
a chave privada do RSA.

\subsection{Geração de Chaves para o RSA} % {{{
A geração das chaves pública e privada utiizadas no RSA é feita em algumas etapas:

Primeiro, são gerados dois números primos de 1024 bits, que chamaremos de $P$ e
$Q$. Um primo é gerado da seguinte forma:

\begin{itemize}
    \item É criada uma \textit{pool} de processos, visto que isso melhora a
        performance do programa.
    \item Cada processo gera um número aleatório ímpar de 1024 bits, usando uma função
        \textit{built-in} do Python -- \verb|secrets.randbits(1024)|.
    \item Os processos testam a primalidade do número gerado com o teste de
        Miller-Rabin. Se o teste der negativo (número com certeza composto), o
        número é incrementado em dois e repetimos a etapa corrente. Caso
        contrário, foi encontrado um número provavelmente primo e essa é a saída
        da função. O algoritmo de Miller-Rabin é executado com 32 iterações, o
        que nos dá uma confiança alta de que o número encontrado realmente é
        primo.
\end{itemize}

O segundo passo é, dado $N = P \cdot Q$, gerar um número $e$ de 16 bits
coprimo com $ \phi(N) = (P-1) \cdot (Q-1) $. Um número aleatório é gerado até que um coprimo seja
encontrado. Com isso, temos a chave pública do RSA $(N, e)$.  

Por último, precisamos encontrar o inverso multiplicativo de $e$ módulo
$\phi(N)$. A fim disso, utilizamos o Algoritmo de Euclides estendido para
resolver a equação diofantina $ed + \lambda \phi(N) = 1$, onde $d$ é o
inverso modular de $e$. Feito isso, temos a chave privada $(N, d)$.

% Geração de chaves }}}
% RSA }}}

\section{Optimal asymmetric encryption padding (OAEP)} % {{{
\label{sec:oaep}
    A cifra "pura" de uma mensagem com RSA apresenta uma série de
    vulnerabilidades, e uma forma de resolvê-las é empregando o esquema OAEP. 

    Na implementação do OAEP foram fixados tamanhos de padding $k_0 = k_1 = 256$ bits.
    Desse modo, são adicionados $k_1$ bits zero à mensagem e gerados $k_0$ bits
    aleatórios, que passam por uma cifra de Feistel de duas rodadas.  

    A função de geração de máscara $MGF1$\citep{mgf1} utiliza como hash base o
    SHA3 de 256 bits. 

    Como a entrada do OAEP é a concatenação $\texttt{hash} | \texttt{chave do
    aes} | \texttt{nonce}$, um bloco de $768$ bits ($96$ bytes), a saída
    tem $768 + 256 + 256 = 1280$ bits ($160$ bytes).


% Optimal asymmetric encryption padding (OAEP) }}}

\section{Advanced Encryption Standard (AES)} % {{{
\label{sec:aes}
Foi implementado o algoritmo AES, também conhecido como Rijndael, para
criptografar a mensagem de entrada. A chave do AES é um bloco de 128 bits
aleatórios, gerados com \verb|secrets.token_bytes(16)|.

A implementação de uma rodada da cifração AES pode ser encontrada na função
\verb|encrypt_block|, que utiliza diversas funções auxiliares:
\begin{itemize}
  \item \verb|sub_bytes|: usa uma SBOX precalculada para substituir os bytes do
      bloco;
  \item \verb|shift_rows|: permuta o bloco por meio de um deslocamento cíclico
      de cada uma de suas linhas;
  \item \verb|mix_columns|: multiplica cada coluna do bloco por um polinômio,
      interpretando a coluna também como um polinômio no corpo finito $GF(2^8)$.
      A multiplicação entre elementos do corpo é implementada com tabelas de
      consulta (definidas em \verb|sign/galois_field.py|), o que simplifica o código e aumenta a performance.
\end{itemize} 

Também foi implementada a função de extensão de chave, que gera blocos de chave
para cada rodada do AES.

O processo de decifração de um bloco também foi escrito, antes do autor perceber que ele não é utilizado no modo contador.

\subsection{Modo contador} % {{{
\label{sec:ctr}
    O modo de operação utilizado com a criptografia AES foi o modo contador (CTR).
    Assim, um \textit{nonce} aleatório de 128 bits é escolhido no início do
    algoritmo e os blocos $nonce$, $nonce+1$, $nonce+2$, $dots$ são
    criptografados com o AES e chave gerada anteriormente. É executado um ou
    exclusivo (xor) entre os blocos da mensagem e os resultantes do $nonce$. O
    resultado disso é mensagem cifrada.

% Modo CTR }}}
    
% Advanced Encryption Standard (AES) }}}

\section{Conclusão} % {{{
\label{sec:conclusao}
    Foi implementado um programa em Python que gera e verifica assinaturas
    digitais baseadas em RSA e OAEP. Além disso, a mensagem a ser enviada é
    cifrada com AES no modo CTR. Esse esquema permite que um usuário verifique
    que uma mensagem foi realmente enviada pelo portador da chave privada do RSA
    e provê uma garantia de integridade.

    Caso esse esquema fosse utilizado para comunicação de mensagens
    \textit{peer-to-peer}, seria possível alterar o programa para usar dois
    pares de chaves de RSA. Desse modo, cada mensagem pode ser cifrada com a
    chave pública de um dos participantes e a chave privada do outro, o que
    impossibilita a decifração por outras pessoas que tenham acesso às mensagens
    cifradas. Como o programa apresentado utiliza apenas a chave de um
    participante, é possível decifrar a mensagem.
    
% Conclusão }}}

% \bibliographystyle{IEEEtran}
\bibliographystyle{unsrtnat}
\bibliography{bib}

\end{document}


